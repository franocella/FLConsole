\chapter{Design}

\section{Software Architecture}

The software architecture of the project is described in this section.

\section{Database Design}

In this section, the database design of the project is presented.
\newpage
\subsection{MongoDB}
\subsubsection{Collections}
\textbf{ExpConfig document example:} \begin{verbatim}
        {
            "id": "example_id",
            "name": "Example Experiment",
            "algorithm": "example_algorithm",
            "strategy": "example_strategy",
            "numClients": 10,
            "stopCondition": "example_condition",
            "threshold": 0.5,
            "parameters": {
                "param1": "value1",
                "param2": "value2",
                "param3": "value3"
            },
            "creationDate": "2024-03-14T00:00:00Z",
            "lastUpdate": "2024-03-14T12:00:00Z"
        }
    \end{verbatim}
\textbf{Experiment document example:} \begin{verbatim}
        {
            "id": "example_id",
            "name": "Example Experiment",
            "expConfigSummary": {
                "id": "exp_config_id",
                "name": "Example Configuration",
                "algorithm": "example_algorithm"
            },
            "creationDate": "2024-03-14T00:00:00Z",
            "lastUpdate": "2024-03-14T12:00:00Z",
            "progressList": [
                {
                "creationDate": "2024-03-14T06:00:00Z",
                "parameters": {
                    "param1": "value1",
                    "param2": "value2"
                },
                "status": "In progress"
                },
                {
                "creationDate": "2024-03-14T09:00:00Z",
                "parameters": {
                    "param1": "value1",
                    "param2": "value2",
                    "param3": "value3"
                },
                "status": "Completed"
                }
            ]
        }
    \end{verbatim}

\newpage
\textbf{User document example:} \begin{verbatim}
        {
        "id": "example_user_id",
        "email": "user@example.com",
        "password": "example_password",
        "creationDate": "2024-03-14T00:00:00Z",
        "configurations": ["config_id1", "config_id2"],
        "experiments": [
            {
            "id": "experiment_summary_id1",
            "name": "Experiment 1",
            "configName": "Configuration 1",
            "creationDate": "2024-03-14T06:00:00Z",
            },
            {
            "id": "experiment_summary_id2",
            "name": "Experiment 2",
            "configName": "Configuration 2",
            "creationDate": "2024-03-14T09:00:00Z",
            "lastUpdate": "2024-03-14T12:00:00Z"
            }
        ],
        "role": "example_role"
    }

    \end{verbatim}
\newpage
\subsection{Erlang Message Handler}

The Erlang message handler design is described in this subsection.

\subsubsection{Message structure}
\begin{itemize}
    \item Error message \begin{verbatim}
        {
            "type": "error",
            "cause": "error_in_collecting_data",
            "timestamp": "2024-03-13T12:34:56"
        }
    \end{verbatim}
    \item Stop message \begin{verbatim}
        {
            "type": "stop",
            "cause": "experiment_finished",
            "timestamp": "2024-03-13T12:34:56"
        }
    \end{verbatim}
    \item Data message \begin{verbatim}
        {   
            "type": "data",
            "parameters": {
                "param1": "value1",
                "param2": "value2"
            },
            "timestamp": "2024-03-13T12:34:56",
            "status": "running"
        }
    \end{verbatim}
\end{itemize}
