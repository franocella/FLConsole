\chapter{Conclusion}
%In this chapter, we summarize the key points of the document and discuss possible future directions for the project.

In this project Federated Learning Web Console is introduced which is a decentralized platform for managing FL experiments with using Java and Erlang 
as primary programming languages, Spring as a Java framework and MongoDB as a database. The primary goal of this project is to implement and provide 
robust system that can coordinate FL experiment with machine learning approach and being in a collaboration with multiple devices while data locality
 is retained. \\
 \\
As a main project architectural structure, the project follows the MVC pattern. Various and functional models are implemented in the project such as 
Configuration, Data Access Object, Services and User Interface with their specific roles and functionalities for serving the application. MongoDB is 
chosen as a document DB database to have scalable and adaptive database. \\
\\
With the help of this architecture FL Web Console project ensures a scalable design and having seamless functionality between different components. 
The user-centric user interface includes essential pages to provide fully functional experience for the user. Those pages are login/signup, user/admin 
dashboard, profile page and experiment details.  
