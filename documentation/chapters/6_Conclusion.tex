\chapter{Conclusion}

%In this chapter, we summarize the key points of the document and discuss possible future directions for the project.

This chapter summarizes and highlights the key points of the document like architecture, implementation details, user interface components and
discusses possible future directions for the project of Federated Learning Web Console.


\subsection{Key Points}
In this project Federated Learning Web Console is introduced which is a centralized platform for managing FL experiments with using Java and Erlang as primary programming languages,
Spring as a Java framework and MongoDB as a database. As a main project architectural structure, the project follows the MVC pattern. Various and functional models are implemented in
the project such as Configuration, Data Access Object, Services and User Interface with their specific roles and functionalities for serving the application. With the help of this
architecture FL Web Console project ensures a scalable design and having seamless functionality between different components. The user interface includes essential pages to provide
fully functional experience for the user. Those pages are login/signup, user/admin dashboard, profile page and experiment details.


