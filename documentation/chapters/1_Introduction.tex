\chapter{Introduction and Project Overview}

\section{Context and Project Objective}

The goal of this project is to develop a system to manage Federated Learning (FL) experiments. FL is a decentralized machine learning approach where multiple devices collaborate to train a shared model while keeping their data locally. The project aims to provide a graphic interface with a web console to run FL experiments, enabling users to monitor their progress and analyze results.

\section{Project Goals}

\subsection{Manage Message Passing}
\begin{itemize}
    \item Implement a communication system between the Java Web Application and Erlang FL Director.\\
    The FL Director is a software that starts experiments among the subscribed devices.
    \item Establish a reliable message passing protocol to exchange information seamlessly.
\end{itemize}

\subsection{Implement a Web Console}
\begin{itemize}
    \item Develop a user-friendly Web Console to initiate and manage FL experiments.
    \item Provide centralized access to experiment statistics for easy monitoring and analysis.
\end{itemize}

\section{Communication Protocol}

To ensure effective communication between the Java Web Application and FL Director, a communication protocol will be defined. It will facilitate the exchange of messages and data in a structured format, ensuring consistency and compatibility across different components of the system. The FL director is an Erlang node, so it's necessary send the data in the proper way.

\section{Data Variety}

FL experiments generate various types of statistics, ranging from model performance metrics to training data distribution. To accommodate this diversity, a flexible data storage mechanism, such as DocumentDB, will be designed. This will allow for efficient storage and retrieval of experiment data while supporting scalability and adaptability.

\section{Concurrent Experiment Execution}

The system will support concurrent execution of multiple FL experiments to maximize resource utilization and efficiency. Java threads and ExecutorService will be used to manage experiment execution concurrently, ensuring optimal performance and resource allocation.

\section{Real-Time Analytics}

Real-time analytics capabilities will be implemented using WebSockets to enable seamless communication between the frontend and backend of the Web Console. This will facilitate real-time monitoring of experiment progress and display of relevant statistics as they become available.

\section{Project Key Points}

\begin{itemize}
    \item Utilize DocumentDB for flexible storage of experiment statistics, allowing for efficient data management and retrieval.
    \item Implement concurrent execution of experiments using Java threads and ExecutorService to optimize resource utilization.
    \item Establish WebSocket communication for real-time data exchange, enabling seamless interaction between the frontend and backend.
    \item Define message formats and outline the structure of Erlang nodes for efficient communication, ensuring reliability and scalability.
    \item Choose a suitable message acknowledgment mechanism to ensure reliable delivery of messages, minimizing the risk of data loss.
\end{itemize}


\section{Architecture and Frameworks}

\begin{itemize}
    \item Adopt the MVC (Model-View-Controller) pattern to structure the Web Console, promoting separation of concerns and maintainability.
    \item Utilize Spring as the Java framework for building the Web Console, leveraging its robust features and ecosystem for rapid development.
    \item Implement WebSockets to enable real-time data communication between the frontend and backend of the Web Console, providing a responsive and interactive user experience.
    \item Employ MongoDB as the database for storing experiment data, benefiting from its flexibility, scalability, and support for complex data structures.
\end{itemize}
