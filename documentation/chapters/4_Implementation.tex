\chapter{Implementation}

\section{Development Environment}

%The development environment used for the project is described in this section.
To be able to have efficient and successful implementation of Federated Learning Web Console Project, having a well-chosen development environment is one of the
most important aspects. In this section, it is specified that the necessary tools, frameworks, and configuration requirements of the project. \\
\begin{itemize}
    \item \textbf{Programming Language:} Java is used for creating a Web Application. Erlang is used for facilitation the development of middleware component and FL director is an Erlang node.
    So that effective communication between the Web Application and the FL director is provided.
    \item \textbf{Frameworks:}Spring is used for Java framework. It ensures to integrate dependencies for WebSocket communication and MongoDB support.
    WebSocket is implemented to provide real-time communication between frontend and backend components.
    \item \textbf{Database Management:} MongoDB is chosen as a database to ensure storing data for experiment statistics and user information.
    \item \textbf{Version Control:} Git is used for version control. It is used to manage the source code of the project. GitHub is used to provide a collaborative development with its version control system. Efficient code management and collaboration is ensured by using repositories which is provided by the platform itself.
    \item \textbf{Integrated Development Environment:} IntelliJ IDEA is used as an IDE. It is a Java integrated development environment for developing computer software. It is developed by JetBrains. It is used to write, compile, and run the code. It also provides a user-friendly interface for developers.
    \item \textbf{Build Automation:} Maven is used for build automation. It is a build automation tool used primarily for Java projects. It is used to manage the project's build, reporting, and documentation from a central piece of information. Maven is used to control project dependencies and build configurations.
    \item \textbf{Testing: } Junit testing is used for testing Java code.
\end{itemize}


\section{Main Modules}

%The main modules of the project are described in this section.
Implementation of the project is structured by diving the project into modules. Each module ensures specific requirements of the project architecture. The modules are:\\
\begin{itemize}
    \item Configuration
    \item Controller
    \item DAO (Data Access Object)
    \item DTO (Data Transfer Object)
    \item Model
    \item Service
    \item Utils
\end{itemize}

\section{Configuration}

%The configuration of the project is described in this section.
Configuration classes of the Federated Learning Web Console project are created to provide responsibilities for configuring different parts of the application such as
logging, execution, HTTP request handling, MVC setup and WebSocket communication. Efficient operation, security and scalability of the system can be ensured by those configuration properties.

\section{Data Access}

%The data access layer of the project is described in this section.
The data access classes are fulfilling the requirements of interacting with the database layers, providing data retrieval, storage, and manipulation. This module includes classes includes CRUD
(create, read, update, delete) operations and query executions. With the Data Access classes such as ExpConfigDao, ExperimentDao, MetricsDao, UserDao the applications guarantee effective operations,
management of experiments and tracking of the progress.


\section{Data Transfer}

%The data transfer mechanisms used in the project are described in this section.
Data Transfer layer contains a ExpConfigSummary, ExperimentSummary and UserSummary classes to ensure the functionality of transferring data structure between different layers and components of the application. With the help of the DTO classes,
related information will be able to be transferred between frontend, backend, and service layers. User information is transferred in a more standardized way
for achieving better communication.

\section{Service}

%The services provided by the project are described in this section.
Service module includes business logic and operations for ensuring the fully functional application. It provides data processing and interaction between different components. Service module includes:\\
\begin{itemize}
    \item Cookie Service is for managing cookie operations such as cookie creation, retrieval, and deletion. The purpose of this service is ensuring session management and personalized user experience.
    \item Experiment Configuration Service is for implementing business logic for experiment configuration includes creation, deletion, retrieval and searching by some parameters.
    \item Experiment Service is for creating operations that are related with experiment like creation, running, deletion, retrieval and searching.
    \item User Service is implemented for ensuring business logic for user-based operations. Those operations include authentication of user, sign up, deletion of account, updating user information and retrieval of the user
    \item Message Service is implemented for managing communication with Erlang node for sending experiment configuration and monitoring the progress.
    \item Metrics Service is created to handle operations of retrieving experiment metrics from the database with the related experiment ID.
\end{itemize}

\section{User Interface}

%The user interface of the project is described in this section.
User Interface module is responsible for providing a user-friendly interface for the users. This module makes application functionalities visible for the end-user. It includes the following components:\\
\begin{itemize}
    \item Login and Sign Up Page: This page is for user authentication and registration. Users can log in to the system by providing their email and password. If the user does not have an account, they can sign up by providing their email, password, and description.
    \item User Dashboard: This page is for displaying the experiments to the user. Users can see experiments and their progress on this page.
    \item Experiment Page: This page is for displaying the details of the experiment. The page shows the details of the experiment and its progress on this page.
    \item Admin Dashboard: This page is for displaying all experiments. Admins can see all experiments and their progress on this page and also it provides creating new experiments for the admin.
    \item Profile Page: The profile page allows users to view and manage their account settings and profile information.
\end{itemize}

\section{Adopted Patterns and Techniques}

%The patterns and techniques adopted in the project are described in this section.

During the implementation of the Federated Learning Web Console project, various patterns and techniques are adopted to ensure the efficiency, scalability, and maintainability of the application. These are some of the used patterns and techniques:\\

\subsection*{Model-View-Controller (MVC) Pattern}
The Federated Learning Web Console project is implemented by following the Model-View-Controller (MVC) pattern. This pattern is used to separate the application into three main components: Model, View, and Controller. The Model represents the data and business logic
of the application, the View represents the presentation layer, and the Controller handles the user input and updates the model and view accordingly. This pattern ensures a clean separation of concerns and makes the application easier to maintain and extend.\\

\subsection*{WebSocket Communication}
WebSocket communication is implemented to provide real-time communication between the frontend and backend. This allows the application to send and receive messages in real-time without the need for polling or long-polling. WebSocket communication is used to update the user interface with the latest data and provide a seamless user experience.\\

\subsection*{Asynchronous Processing}
Asynchronous processing techniques like Java threads and ExecutorService are e used to manage concurrent execution of experiments and other tasks. This allows the application to handle multiple requests and tasks simultaneously and improve performance and scalability.\\

\subsection*{Message Passing Protocol}
To achieve seamless communication between Erlang FL director and Java web application, a customized and well specified message passing protocol is defined. This protocol guarantees the reliable and well-defined exchange of messages and data.
